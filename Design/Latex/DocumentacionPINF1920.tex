%%\documentclass[a4paper,12pt,oneside]{llncs}
\documentclass[12pt,letterpaper]{report}
\usepackage[right=2cm,left=3cm,top=2cm,bottom=2cm,headsep=0cm]{geometry}

%%%%%%%%%%%%%%%%%%%%%%%%%%%%%%%%%%%%%%%%%%%%%%%%%%%%%%%%%%%
%% Juego de caracteres usado en el archivo fuente: UTF-8
\usepackage{ucs}
\usepackage[utf8x]{inputenc}

%%%%%%%%%%%%%%%%%%%%%%%%%%%%%%%%%%%%%%%%%%%%%%%%%%%%%%%%%%%
%% Juego de caracteres usado en la salida dvi
%% Otra posibilidad: \usepackage{t1enc}
\usepackage[T1]{fontenc}

%%%%%%%%%%%%%%%%%%%%%%%%%%%%%%%%%%%%%%%%%%%%%%%%%%%%%%%%%%%
%% Ajusta maergenes para a4
%\usepackage{a4wide}

%%%%%%%%%%%%%%%%%%%%%%%%%%%%%%%%%%%%%%%%%%%%%%%%%%%%%%%%%%%
%% Uso fuente postscript times, para que los ps y pdf queden y pequeños...
\usepackage{times}

%%%%%%%%%%%%%%%%%%%%%%%%%%%%%%%%%%%%%%%%%%%%%%%%%%%%%%%%%%%
%% Posibilidad de hipertexto (especialmente en pdf)
%\usepackage{hyperref}
\usepackage[bookmarks = true, colorlinks=true, linkcolor = black, citecolor = black, menucolor = black, urlcolor = black]{hyperref}

%%%%%%%%%%%%%%%%%%%%%%%%%%%%%%%%%%%%%%%%%%%%%%%%%%%%%%%%%%%
%% Graficos 
\usepackage{graphics,graphicx}

%%%%%%%%%%%%%%%%%%%%%%%%%%%%%%%%%%%%%%%%%%%%%%%%%%%%%%%%%%%
%% Ciertos caracteres "raros"...
\usepackage{latexsym}

%%%%%%%%%%%%%%%%%%%%%%%%%%%%%%%%%%%%%%%%%%%%%%%%%%%%%%%%%%%
%% Matematicas aun más fuertes (american math dociety)
\usepackage{amsmath}

%%%%%%%%%%%%%%%%%%%%%%%%%%%%%%%%%%%%%%%%%%%%%%%%%%%%%%%%%%%
\usepackage{multirow} % para las tablas
\usepackage[spanish,es-tabla]{babel}

%%%%%%%%%%%%%%%%%%%%%%%%%%%%%%%%%%%%%%%%%%%%%%%%%%%%%%%%%%%
%% Fuentes matematicas lo mas compatibles posibles con postscript (times)
%% (Esto no funciona para todos los simbolos pero reduce mucho el tamaño del
%% pdf si hay muchas matamaticas....
\usepackage{mathptm}

%%% VARIOS:
\usepackage{slashbox}
\usepackage{verbatim}
\usepackage{array}
\usepackage{listings}
\usepackage{multirow}

%% MARCA DE AGUA
%% Este package de "draft copy" NO funciona con pdflatex
%%\usepackage{draftcopy}
%% Este package de "draft copy" SI funciona con pdflatex
%%%\usepackage{pdfdraftcopy}
%%%%%%%%%%%%%%%%%%%%%%%%%%%%%%%%%%%%%%%%%%%%%%%%%%%%%%%%%%%
%% Indenteacion en español...
\usepackage[spanish]{babel}
\usepackage{listings}
% Para escribir código en C
% \begin{lstlisting}[language=C]
% #include <stdio.h>
% int main(int argc, char* argv[]) {
% puts("Hola mundo!");
% }
% \end{lstlisting}


\title{SendUCA}
\author{Grupo 1 PINF}

\begin{document}
	\maketitle
	\thispagestyle{empty}
	\newpage
	\tableofcontents
	%%\listoftables
	%%\newpage
	
	%%\listoffigures
	%%\newpage
	
	%%%% REAL WORK BEGINS HERE:
	
	%%Configuracion del paquete listings
	\lstset{language=bash, numbers=left, numberstyle=\tiny, numbersep=10pt, firstnumber=1, stepnumber=1}
	
\chapter{Prolegómeno}		
\section{Introducción}
	\subsection{Motivación}
	\subsection{Descripción del sistema actual}
	\subsection{Objetivos y alcance del proyecto}
	\subsection{Organización del documento}

\section{Planificación}
	\subsection{Metodología de desarrollo}
	\subsection{Planificación}

\chapter{Desarrollo}
\section{Análisis de requisitos}

	\subsection{Objetivos del sistema}
		\noindent Controlar el acceso al sistema\\
		Descripción: El sistema deberá gestionar el acceso a la aplicación, restringiendo el mismo a personas registradas en el sistema y autentificando las credenciales del usuario.
		\\~\\
		Realizar votaciones\\
		Descripción: El sistema deberá permitir la creación y la participación de los usuarios del sistema en votaciones, en base al cumplimiento de unas pre condiciones específicas.
		\\~\\
		Realizar Elecciones \\
		Descripción: 
		El sistema deberá permitir la creación y la participación de los usuarios del sistema en elecciones, en base al cumplimiento de unas pre condiciones específicas.
		\\~\\ 
		Realizar Consultas\\
		Descripción: El sistema deberá permitir configurar la creación de una votación / elección como una consulta.
		\\~\\  
		Mostrar Estadísticas\\
		Descripción: El sistema deberá ser capaz de mostrar estadísticas referentes a cada votación / elección / consulta que se realice.\\   
		
	\subsection{Catálogo de actores}
		\noindent Administrador\\
		Descripción: Se encarga de la asignación de permisos al resto de usuarios, velar por la integridad del sistema, etc. Este rol lo pueden desempeñar una persona o varias, con distintas claves de acceso.
		\\~\\
		Elector\\
		Descripción: Cada uno de los usuarios incluidos en el censo de una elección/votación y que tiene derecho al voto en la misma.
		\\~\\
		Secretario\\
		Descripción: Son los encargados de dar de alta los procesos electorales, fijando las características de los mismos como censo, calendarios, etc.
		\\~\\
		Secretario general\\
		Descripción: En algunos procesos electorales los secretarios[ACT-003] pueden delegar parte de sus funciones en los secretarios delegados. También el administrador[ACT-001] puede crear esta figura, en el caso de ausencia de los secretarios.
		\\~\\
		
	\subsection{Requisitos funcionales}
	
		\begin{table}[htp]
			\resizebox{.8\textwidth}{!}{
			\begin{tabular}{|c|c|}
				\hline
				UC-0001          & Iniciar Sesión                                                                                                                                                                                                                                                                                                                                                                                                                \\ \hline
				Descripción      & \begin{tabular}[l]{@{}l@{}}El sistema deberá comportarse tal como se describe \\ en el siguiente caso de uso cuando algún usuario \\ desee iniciar sesión para acceder al sistema.\end{tabular}                                                                                                                                                                                                                               \\ \hline
				Precondición     & Usuario sin iniciar sesión.                                                                                                                                                                                                                                                                                                                                                                                                   \\ \hline
				Secuencia normal & \begin{tabular}[l]{@{}l@{}}1.El sistema muestra la pantalla para iniciar sesión.\\ 2.El usuario introduce sus datos en la pantalla.\\ 3.El sistema verifica que los datos son correctos y muestra \\ la pantalla principal asociada al rol del usuario.\end{tabular}                                                                                                                                                          \\ \hline
				Postcondición    & Usuario con sesión iniciada.                                                                                                                                                                                                                                                                                                                                                                                                  \\ \hline
				Excepciones      & \begin{tabular}[l]{@{}l@{}}1.Si el usuario ya no desea loguearse, el usuario cancela\\  el inicio de sesión, a continuación este caso de uso queda sin efecto.\\ 2.Si el usuario no existe, el sistema mostrará un mensaje\\  de error, a continuación este caso de uso queda sin efecto.\\ 3.Si los datos son incorrectos, el sistema mostrará un\\ mensaje de error, a continuación este caso de uso continúa.\end{tabular} \\ \hline
				Comentarios      & Ninguno                                                                                                                                                                                                                                                                                                                                                                                                                       \\ \hline
			\end{tabular}}
		\end{table}
		\hspace{2em}
		\begin{table}[htp]
			\resizebox{.8\textwidth}{!}{
			\begin{tabular}{|c|c|}
				\hline
				UC-0002          & Crear Votación                                                                                                                                                                                                                                                                                            \\ \hline
				Descripción      & \begin{tabular}[l]{@{}l@{}}El sistema deberá comportarse tal como se describe \\ en el siguiente caso de uso cuando algún usuario con los permisos \\ oportunos desee crear una votación.\end{tabular}                                                                                                    \\ \hline
				Precondición     & Usuario previamente logueado.                                                                                                                                                                                                                                                                             \\ \hline
				Secuencia normal & \begin{tabular}[l]{@{}l@{}}1.El sistema muestra la pantalla para crear una votación.\\ 2.El usuario establece los datos de creación\\  asociados a la votación.\\ 3.El sistema verifica que los datos son correctos y muestra un mensaje de\\  éxito indicando que la votación se ha creado.\end{tabular} \\ \hline
				Postcondición    & Votación generada exitosamente.                                                                                                                                                                                                                                                                           \\ \hline
				Excepciones      & \begin{tabular}[l]{@{}l@{}}1.Si el usuario ya no desea crear la votación, el usuario cancela \\ la creación, a continuación este caso de uso queda sin efecto.\\ 2.Si los datos son incorrectos, el sistema mostrará un mensaje de \\ error, a continuación este caso de uso continúa.\end{tabular}       \\ \hline
				Comentarios      & Ninguno                                                                                                                                                                                                                                                                                                   \\ \hline
			\end{tabular}}
		\end{table}
		\hspace{2em}
		\begin{table}[htp]
			\resizebox{.8\textwidth}{!}{
			\begin{tabular}{|c|c|}
				\hline
				UC-0003          & Crear Elección                                                                                                                                                                                                                                                                                         \\ \hline
				Descripción      & \begin{tabular}[l]{@{}l@{}}El sistema deberá comportarse tal como se describe \\ en el siguiente caso de uso cuando algún usuario con los\\  permisos oportunos desee crear una elección.\end{tabular}                                                                                                 \\ \hline
				Precondición     & Usuario previamente logueado.                                                                                                                                                                                                                                                                          \\ \hline
				Secuencia normal & \begin{tabular}[l]{@{}l@{}}1.El sistema muestra la pantalla para crear una elección.\\ 2.El usuario establece los datos de creación asociados a la elección.\\ 3.El sistema verifica que los datos son correctos y muestra un\\  mensaje de éxito indicando que la elección se ha creado.\end{tabular} \\ \hline
				Postcondición    & Elección generada exitosamente.                                                                                                                                                                                                                                                                        \\ \hline
				Excepciones      & \begin{tabular}[l]{@{}l@{}}1.Si el usuario ya no desea crear la elección, el usuario cancela \\ la creación, a continuación este caso de uso queda sin efecto.\\ 2.Si los datos son incorrectos, el sistema mostrará un mensaje de \\ error, a continuación este caso de uso continúa.\end{tabular}    \\ \hline
				Comentarios      & Ninguno                                                                                                                                                                                                                                                                                                \\ \hline
			\end{tabular}}
		\end{table}
		\hspace{2em}
		\begin{table}[htp]
			\resizebox{.8\textwidth}{!}{
			\begin{tabular}{|c|c|}
				\hline
				UC-0004          & Crear Consulta                                                                                                                                                                                             \\ \hline
				Descripción      & \begin{tabular}[l]{@{}l@{}}El sistema deberá comportarse tal como se describe en \\ el siguiente caso de uso cuando algún usuario con los permisos \\ oportunos desee crear una consulta.\end{tabular}     \\ \hline
				Precondición     & \begin{tabular}[l]{@{}l@{}}Usuario previamente logueado y realizando la creación \\ de una votación / elección.\end{tabular}                                                                               \\ \hline
				Secuencia normal & \begin{tabular}[l]{@{}l@{}}1.El sistema muestra la opción para la generación\\  de una consulta.\\ 2.El usuario selecciona la opción de generar una\\  votación / elección como una consulta.\end{tabular} \\ \hline
				Postcondición    & Votación / Elección generada exitósamente como una Consulta.                                                                                                                                               \\ \hline
				Excepciones      & \begin{tabular}[l]{@{}l@{}}Si el usuario ya no desea crear la votación / elección como\\  una consulta, el usuario desmarca la opción,  a continuación \\ \\ este caso de uso continúa.\end{tabular}       \\ \hline
				Comentarios      & \begin{tabular}[l]{@{}l@{}}En caso de cumplirse la excepción, se creará una votación\\  o una elección, definidos sus casos de uso\\  respectivamente en UC-0003 y UC-0004\\ .\end{tabular}                \\ \hline
			\end{tabular}}
		\end{table}
		\hspace{2em}
		\begin{table}[htp]
			\resizebox{.8\textwidth}{!}{
			\begin{tabular}{|c|c|}
				\hline
				UC-0005          & Participar en Votación Simple                                                                                                                                                                                                                                                                                                   \\ \hline
				Descripción      & \begin{tabular}[l]{@{}l@{}}El sistema deberá comportarse tal como se describe en el \\ siguiente caso de uso cuando algún usuario desee participar\\  en una votación de tipo simple.\end{tabular}                                                                                                                              \\ \hline
				Precondición     & \begin{tabular}[l]{@{}l@{}}Usuario logueado y con permisos para participar en una\\  votación de tipo simple.\end{tabular}                                                                                                                                                                                                      \\ \hline
				Secuencia normal & \begin{tabular}[l]{@{}l@{}}1.El sistema muestra la pantalla para participar en la votación.\\ 2.El sistema muestra la pregunta, y las opciones de la votación\\  (a favor, en contra, abstención).\\ 3.El usuario lee la pregunta, y selecciona la opción deseada.\\ 4.El sistema recoge los datos de la votación.\end{tabular} \\ \hline
				Postcondición    & Usuario participa exitosamente en la votación simple.                                                                                                                                                                                                                                                                           \\ \hline
				Excepciones      & \begin{tabular}[l]{@{}l@{}}1.Si el usuario ya no desea votar, el usuario cancela su \\ votación, a continuación este caso de uso queda sin efecto.\\ 2.Si el usuario no selecciona una opción, el sistema mostrará un\\  mensaje de error, a continuación este caso de uso continúa.\end{tabular}                               \\ \hline
				Comentarios      & Ninguno.                                                                                                                                                                                                                                                                                                                        \\ \hline
			\end{tabular}}
		\end{table}
		\hspace{2em}
		\begin{table}[htp]
			\resizebox{.8\textwidth}{!}{
			\begin{tabular}{|c|c|}
				\hline
				UC-0006          & Participar en Votación Compleja                                                                                                                                                                                                                                                                                     \\ \hline
				Descripción      & \begin{tabular}[l]{@{}l@{}}El sistema deberá comportarse tal como se describe en el \\ siguiente caso de uso cuando algún usuario desee participar\\  en una votación de tipo compleja.\end{tabular}                                                                                                                \\ \hline
				Precondición     & \begin{tabular}[l]{@{}l@{}}Usuario logueado y con permisos para participar en\\ una votación de tipo compleja.\end{tabular}                                                                                                                                                                                         \\ \hline
				Secuencia normal & \begin{tabular}[l]{@{}l@{}}1.El sistema muestra la pantalla para participar en la votación.\\ 2.El sistema muestra la pregunta, y las opciones de la votación (\\ múltiples opciones).\\ 3.El usuario lee la pregunta, y selecciona la opción deseada.\\ 4.El sistema recoge los datos de la votación.\end{tabular} \\ \hline
				Postcondición    & Usuario participa exitosamente en la votación compleja.                                                                                                                                                                                                                                                             \\ \hline
				Excepciones      & \begin{tabular}[l]{@{}l@{}}1.Si el usuario ya no desea votar, el usuario cancela su \\ votación, a continuación este caso de uso queda sin efecto.\\ 2.Si el usuario no selecciona una opción, el sistema mostrará un\\  mensaje de error, a continuación este caso de uso continúa.\end{tabular}                   \\ \hline
				Comentarios      & Ninguno.                                                                                                                                                                                                                                                                                                            \\ \hline
			\end{tabular}}
		\end{table}
		\hspace{2em}
		\begin{table}[htp]
			\resizebox{.8\textwidth}{!}{
			\begin{tabular}{|c|c|}
				\hline
				UC-0007          & Participar en Elecciones Unipersonales                                                                                                                                                                                                                                                            \\ \hline
				Descripción      & \begin{tabular}[l]{@{}l@{}}El sistema deberá comportarse tal como se describe en el\\  siguiente caso de uso cuando algún usuario desee\\  participar en una elección a cargos unipersonales.\end{tabular}                                                                                        \\ \hline
				Precondición     & \begin{tabular}[l]{@{}l@{}}Usuario logueado y con permisos para participar en una \\ \\ elección a cargos unipersonale.\end{tabular}                                                                                                                                                              \\ \hline
				Secuencia normal & \begin{tabular}[l]{@{}l@{}}1.El sistema muestra la pantalla para participar en la elección.\\ 2.El sistema muestra los candidatos elegibles.\\ 3.El usuario lee la información de los candidatos, y selecciona\\  la opción deseada.\\ 4.El sistema recoge los datos de la elección.\end{tabular} \\ \hline
				Postcondición    & Usuario participa exitosamente en la elección a cargos unipersonales.                                                                                                                                                                                                                             \\ \hline
				Excepciones      & \begin{tabular}[l]{@{}l@{}}1.Si el usuario ya no desea votar, el usuario cancela su \\ elección, a continuación este caso de uso queda sin efecto.\\ 2.Si el usuario no selecciona una opción, el sistema mostrará un \\ mensaje de error, a continuación este caso de uso continúa.\end{tabular} \\ \hline
				Comentarios      & Ninguno.                                                                                                                                                                                                                                                                                          \\ \hline
			\end{tabular}}
		\end{table}
		\hspace{2em}
		\begin{table}[htp]
			\resizebox{.8\textwidth}{!}{
			\begin{tabular}{|c|c|}
				\hline
				UC-0008          & Participar en Elecciones por Grupos                                                                                                                                                                                                                                                                  \\ \hline
				Descripción      & \begin{tabular}[l]{@{}l@{}}El sistema deberá comportarse tal como se describe en el\\  siguiente caso de uso cuando algún usuario desee\\  participar en una elección por grupos.\end{tabular}                                                                                                       \\ \hline
				Precondición     & \begin{tabular}[l]{@{}l@{}}Usuario logueado y con permisos para participar \\ en una elección por grupos.\end{tabular}                                                                                                                                                                               \\ \hline
				Secuencia normal & \begin{tabular}[l]{@{}l@{}}1.El sistema muestra la pantalla para participar en la \\ elección.\\ 2.El sistema muestra los candidatos elegibles.\\ 3.El usuario lee la información de los candidatos, y selecciona\\  la opción deseada.\\ 4.El sistema recoge los datos de la elección.\end{tabular} \\ \hline
				Postcondición    & Usuario participa exitosamente en la elección por grupos.                                                                                                                                                                                                                                            \\ \hline
				Excepciones      & \begin{tabular}[l]{@{}l@{}}1.Si el usuario ya no desea votar, el usuario cancela su elección, \\ a continuación este caso de uso queda sin efecto.\\ 2.Si el usuario no selecciona una opción, el sistema\\  mostrará un mensaje de error, a \\ continuación este caso de uso continúa.\end{tabular} \\ \hline
				Comentarios      & Ninguno.                                                                                                                                                                                                                                                                                             \\ \hline
			\end{tabular}}
		\end{table}
		\hspace{2em}
		\begin{table}[htp]
			\resizebox{.8\textwidth}{!}{
			\begin{tabular}{|c|c|}
				\hline
				UC-0009          & Mostrar Estadísticas                                                                                                                                                                                                         \\ \hline
				Descripción      & \begin{tabular}[l]{@{}l@{}}El sistema deberá comportarse tal como se describe en el siguiente \\ caso de uso cuando algún usuario desee ver las estadísticas asociadas\\  a una votación / elección / consulta.\end{tabular} \\ \hline
				Precondición     & Usuario logueado y votación / elección / consulta previamente concluída.                                                                                                                                                     \\ \hline
				Secuencia normal & \begin{tabular}[l]{@{}l@{}}1.El usuario selecciona la opción de mostrar estadísticas.\\ 2.El sistema muestra los datos estadísticos asociados a la\\  votación / elección / consulta.\end{tabular}                           \\ \hline
				Postcondición    & Estadísticas visualizadas exitosamente                                                                                                                                                                                       \\ \hline
				Excepciones      & \begin{tabular}[l]{@{}l@{}}Si no se ha realizado ninguna votación / elección / consulta, el\\  sistema mostrará un mensaje de error, a continuación este caso \\ de uso queda sin efecto.\end{tabular}                       \\ \hline
				Comentarios      & Ninguno.                                                                                                                                                                                                                     \\ \hline
			\end{tabular}}
		\end{table}
	
	\subsection{Modelos estáticos del sistema}

		\subsubsection{Modelo Estático de los Procesos Electorales.}
			\begin{figure}[htp]
				\centering
				\includegraphics[width=0.9\linewidth]{img/MEProcesosElectorales}
				\caption{}
				\label{fig:meprocesoselectorales}
			\end{figure}
		\newpage
		
		\subsubsection{Modelo Estático de la jerarquía de usuarios del Sistema}
			\begin{figure}[htp]
				\centering
				\includegraphics[width=0.9\linewidth]{img/MEJerarquiaUsuarios}
				\caption{}
				\label{fig:meprocesoselectorales}
			\end{figure}
		\newpage	
	\subsection{Requisitos de información}
		
	\newpage
	\subsection{Requisitos no funcionales}
		\begin{table}[htp]
			\resizebox{.8\textwidth}{!}{
			\begin{tabular}{|c|c|c|}
				\hline
				ID   & Requerimiento                  & Descripción                                                                                                                                                                                                                                                                   \\ \hline
				RNF1 & Seguridad                      & \begin{tabular}[c]{@{}c@{}}Mecanismos que permiten la protección\\  de los datos y la estabilidad del sistema\\ frente a cualquier tipo de ataque. \\ Garantizando además el secreto de \\ voto, preservando siempre el anonimato de la\\  elección del elector.\end{tabular} \\ \hline
				RNF2 & Accesibilidad                  & \begin{tabular}[c]{@{}c@{}}Se adapta a las posibles diversidades funcionales \\ de cualquier individuo.\end{tabular}                                                                                                                                                          \\ \hline
				RNF3 & Mantenibilidad / Escalabilidad & \begin{tabular}[c]{@{}c@{}}Propiedad que indica la capacidad de reaccionar\\ y adaptarse a la carga en el sistema sin perder calidad.\end{tabular}                                                                                                                            \\ \hline
				RNF4 & Portabilidad                   & \begin{tabular}[c]{@{}c@{}}Capacidad de acceder al servicio desde cualquier \\ dispositivo con acceso a un navegador web\end{tabular}                                                                                                                                         \\ \hline
				RNF5 & Usabilidad                     & Claridad en los procesos y uso de la aplicación.                                                                                                                                                                                                                              \\ \hline
			\end{tabular}}
		\end{table}
		
	\newpage
	\subsection{Reglas de negocio}
	
		\begin{table}[htp]
			\resizebox{.8\textwidth}{!}{
			\begin{tabular}{|c|l|}
				\hline
				RN-002              & Pertenencia a grupos                                \\ \hline
				Descripción         & Cada usuario puede pertenecer a uno o varios grupos \\ \hline
				Reglas relacionadas &                                                     \\ \hline
				Última modificación & 12/12/2019                                          \\ \hline
			\end{tabular}}
		\end{table}
	
		\begin{table}[htp]
			\resizebox{.8\textwidth}{!}{
			\begin{tabular}{|c|l|}
				\hline
				RN-002              & Miembro de mesa electoral                                                                                                      \\ \hline
				Descripción         & \begin{tabular}[c]{@{}l@{}}A efectos de realización digital del recuento \\ debe ser introducida obligatoriamente\end{tabular} \\ \hline
				Reglas relacionadas & RN-003                                                                                                                         \\ \hline
				Última modificación & 12/12/2019                                                                                                                     \\ \hline
			\end{tabular}}
		\end{table}
	
		\begin{table}[htp]
			\resizebox{.8\textwidth}{!}{
			\begin{tabular}{|c|c|}
				\hline
				RN-003              & Rol de elector                                                                                                                            \\ \hline
				Descripción         & \begin{tabular}[c]{@{}c@{}}El rol de elector de un usuario viene dado \\por la aparición del mismo en un determinado censo.\end{tabular} \\ \hline
				Reglas relacionadas &                                                                                                                                           \\ \hline
				Última modificación & 12/12/2019                                                                                                                                \\ \hline
			\end{tabular}}
		\end{table}
	
		\begin{table}[htp]
			\resizebox{.8\textwidth}{!}{
			\begin{tabular}{|c|l|}
				\hline
				RN-004              & Información del censo                                                                                                                   \\ \hline
				Descripción         & \begin{tabular}[c]{@{}l@{}}Un fichero de censo debe contener el nombre de cada elector, \\ DNI e identificador de usuario.\end{tabular} \\ \hline
				Reglas relacionadas & RN-003                                                                                                                                  \\ \hline
				Última modificación & 12/12/2019                                                                                                                              \\ \hline
			\end{tabular}}
		\end{table}
		
	\newpage	
	\subsection{Estudio de alternativas tecnológicas}
		Ninguna por el momento.
\section{Diseño de Sistema}
	\subsection{Arquitectura}
	\subsection{Tipos y por qué}
	\subsection{Diseño}
\section{Implementación del sistema}
	\subsection{Entorno tecnológico}
	\subsection{Código fuente}
	\subsection{Calidad del código}
\section{Pruebas del sistema}
	\subsection{Pruebas unitarias}
	\subsection{Pruebas de integración}
	\subsection{Pruebas de sistema}
		\subsubsection{Pruebas funcionales}
		\subsubsection{Pruebas no funcionales}
	\subsection{Pruebas de aceptación}
\chapter{Epílogo}
\section{Manual de Usuario}
	\subsection{Introducción}
	\subsection{Características}
	\subsection{Requisitos previos}
	\subsection{Utilización}

\section{Manual de instalación y explotación}
	\subsection{Introducción}
	\subsection{Requisitos previos}
	\subsection{Inventario de componentes}

\section{Conclusiones}
	\subsection{Objetivos}
	\subsection{Lecciones aprendidas}
	\subsection{Trabajo futuro}

\section{bibliografías}

\section{Licencias}

\section{Índice de figuras}

\section{Índice de tablas}

\end{document}